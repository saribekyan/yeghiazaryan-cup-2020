\hypertarget{info-sheet}{}
\paragraph{General information about interactive and stdin/stdout problems}
\begin{itemize}
    \item Time limit for all interactive or stdin/stdout problems is 2 seconds and memory limit is 128MB.
    \item All input/output should be done from standard input and output.
    You can find examples of standard input/output in common languages \href{http://ejudge.rau.am/help.html}{here}.
\end{itemize}

\paragraph{General remarks about interactive problems}
\hypertarget{info-interactive}{}
\begin{enumerate}
    \item You must print a new line after each interaction;
    \item You must flush the output stream after each interaction:
    \begin{itemize}
        \item In C or C++: \texttt{fflush(stdout);}
        \item In Java: \texttt{System.out.flush();}
        \item In Python: \texttt{sys.stdout.flush()}
        \item In C\#: \texttt{Console.Out.Flush();}
    \end{itemize}
    
    \item If your program receives EOF (end-of-file) condition on the standard input, it \emph{must} exit immediately with exit code 0.
    Failure to comply with this requirement may result in Time Limit Exceeded error instead of other verdicts AC/WA/PE.
    
    \item Typical issues with interactive problems are
    \begin{itemize}
        \item Wrong Answer -- usually means that your program followed the interaction protocol but the answer or the intermediate steps are wrong.
        
        \item (Wall) Time Limit Exceeded -- this means that due to your program's execution, the interaction is not progressing.
        This can happen 
        \begin{itemize}
            \item if your program is expecting an input from the jury’s program by mistake.
            Most commonly this happens when the jury's program has detected an error and quit, but your program is waiting for an input.
            In this case, you will get the TLE verdict, instead of the actual WA/PE error.
            See point (3) above;
            \item if your program has not provided the necessary output for the jury’s program to respond.
            Most often this is because you have not flushed the output stream.
            See point (2) above.
        \end{itemize}
        
        \item Presentation Error -- usually means that your program did not follow the interaction protocol correctly and the jury’s interacting protocol is not able to test it.
        
        \item Runtime error -- usually a mistake in your program that makes your program crash during the execution.
    \end{itemize}
\end{enumerate}
