\paragraph{General remarks about interactive problems.}
\begin{itemize}
    \item You must print a new line after each interaction;
    \item You must flush the output stream after each interaction:
    \begin{itemize}
        \item In C or C++: \texttt{fflush(stdout);}
        \item In Java: \texttt{System.out.flush();}
        \item In Python: \texttt{sys.stdout.flush()}
        \item In C\#: \texttt{Console.Out.Flush();}
    \end{itemize}
    \item Typical issues with interactive problems are
    \begin{itemize}
        \item Wrong Answer -- usually means that your program followed the interaction protocol but the answer or the intermediate steps are wrong.
        
        \item Presentation Error -- usually means that your program did not follow the interaction protocol correctly and the jury’s interacting protocol is not able to test it.
        Note that this may happen if your output does not satisfy the required upper/lower limits of numbers.
        
        \item Wall Time Limit Exceeded -- this means that your program is not following the protocol in such a way that the interaction is not progressing.
        This can happen if your program is expecting an input from the jury’s program by mistake; or if your program has not provided the necessary output for the jury’s program to respond.
        The latter can happen if you do not flush the output stream.
        
        \item Runtime error -- usually a mistake in your program that makes your program crash during the execution.
    \end{itemize}
\end{itemize}
