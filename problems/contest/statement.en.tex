\problemname[stdin/stdout]{Contest}
\newcommand{\compname}{NAME}
Hayk and Eduard like preparing programming competitions together and are now working on the problem set for \compname{} - the latest addition to the list of major programming contests in the world.\htodo{fill}
They have exactly $T$ days until the competition.
Hayk has created $N$ problems, and he needs to convince Eduard that his problems are good enough for \compname.

To finalise the problem set, Hayk and Eduard meet once a day until the day of the competition ($T$ times in total).
For each meeting, Hayk decides to focus only on one of his problems and 
tries to convince Eduard to include that problem in \compname{}.
If Eduard sees problem $i$ for the first time, he will agree to include it
the competition with probability $p_i$.
However, Eduard does not like repeating things, and if Hayk proposes problem $i$
for the second time later, Eduard will agree with him with probability $f \cdot p_i$.
In general, the probability that Eduard accepts problem $i$ on the $k$th proposal is $f^{k-1}\cdot p_i$, i.e., it decreases by factor $f$ every time problem $i$ is proposed by Hayk.\htodo{say that only for this problem, others do not decrease}

Not all problems are created equal.
Hayk has given a score $S_i$ for problem $i$, which determines how much he likes that problem.
He wants to increase the total score $S$ of his problems at \compname{}.
Hayk knows Eduard very well and has calculated the values $p_i$ and $f$ exactly.
Help Hayk to compute the largest expected value of $S$ that he can have.
