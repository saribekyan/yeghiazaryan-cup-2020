% from NCPC 2008, Problem F - Fixing the Bugs
\problemname[stdin/stdout]{Contest}
Hayk and Eduard like preparing programming competitions together and are now working on the problem set for CEC -- Coronavirus End Cup, the largest programming contest in history.
They have exactly $T$ days until the competition.
Hayk has created $N$ problems, and he needs to convince Eduard that his problems are good enough for CEC.

To finalise the problem set, Hayk and Eduard meet once a day until the day of the competition ($T$ times in total).
For each meeting, Hayk focuses only on one of his problems and 
tries to convince Eduard to include that problem in CEC.
If Eduard sees problem $i$ for the first time, he will agree to include it in CEC with probability $p_i$, and reject it with probability $1 - p_i$.
If it is rejected, Hayk can suggest the same problem on another meeting later.
However, Eduard does not like to repeat things, and if Hayk proposes problem $i$
for the second time, Eduard will agree with him with probability only $f \cdot p_i$.
In general, the probability that Eduard accepts problem $i$ on its $k$th proposal is $f^{k-1}\cdot p_i$, i.e., it decreases by factor $f$ every time problem $i$ is proposed by Hayk.
Notice that the probability decreases only for problem $i$, and stays the same for the rest.

Not all problems are created equal.
Hayk has assigned a score $S_i$ to problem $i$, which determines how much he likes that problem.
He wants to increase the total score $S$ of his problems at CEC.
Hayk knows Eduard very well and has calculated the values $p_i$ and $f$ exactly.
Help Hayk to compute the largest expected value of $S$ that he can have.

\section*{Input}
The first line of the input contains the integer $N (0 \leq N \leq 10)$, the number of Hayk's problems; the integer $T (0 \leq T \leq 100)$, the number of days until CEC; and the real number $f (0 \leq f \leq 1)$, the factor by which Eduard's mood changes.

$N$ lines follow.
The $i$th line contains the values $p_i (0 \leq p_i \leq 1)$, the probability that
Eduard agrees problem $i$ the first time he sees it;
and $S_i (0 \leq S_i \leq 10000)$, the score of problem $i$.

\section*{Output}
The only line of the output must contain the largest expected value of $S$, with at least $5$ digits after the decimal point.
