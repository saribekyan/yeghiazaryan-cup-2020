\problemname{Earthovirus}
It is year 2200.
Not only we have survived but you are on Mars and your mission is to isolate Mars from the only dangerous virus left in the universe: the Earthovirus.
For that you have to compare the DNA of the suspected virus against the DNA of Earthovirus.

DNAs of viruses can be represented as strings composed of letters A, G, C, T.
The complete DNA of Earthovirus is given by a string $E$ of length $n$.
Due to the radiation, the DNAs of all other viruses in the universe are uniformly random strings of arbitrary length.\footnote{Each character of the string is one of the four possibilities, with probability $1/4$, independently from all other characters.}
Thus, if the DNA of the suspected virus is represented by the string $X$, then either $X = E$ or $X$ is a random string composed of characters A, G, C, T.

When the laboratory processes the suspected virus, its DNA $X$ breaks in some places. %, creating multiple contiguous parts of the DNA.
The parts that have less than $100$ characters are very small and get lost.
So, only some disjoint substrings $Z_1, Z_2, \dots, Z_k$ of test DNA $X$ remain and each substring $Z_i$ is at least $100$ characters long.
The laboratory guarantees that in total at least $n / 2$ characters remain in those substrings.

You can give the laboratory a string $S$ representing a DNA code, and they will quickly check whether $S$ is a substring of some $Z_i$.
You need to write a program that asks no more than such $100$ questions and determines whether the test virus is an Earthovirus or no.

\paragraph{The interaction between your program and the laboratory.}
This is an interactive problem.
First, your program will need to read from the standard input the string $E$.
After this your program needs to write in the standard output a substring $S$, to be checked by the laboratory.
If $S$ is a substring of $Z_i$ for some $i = 1, \dots, k$, then your program will get an answer ``Yes'' in the standard input, otherwise it will receive an answer ``No''.
If at any point your program has determined the type of the virus, it has to print either ``Earthovirus'' or ``Randovirus'', depending on which it is.
After this, your program must stop.
Finally, do not forget to flush after each output.

\paragraph{Limits.}
It is guaranteed that $1 \leq n \leq 10^5$.
You are not allowed to ask more than $100$ questions, and the total number of characters in the DNA-strings that you ask must not exceed $10^6$.
All your questions must be strings representing a valid DNA.

\paragraph{Sample executions.}
The sample executions below are for demonstration only and are created manually.
Also, it is assumed that no substring of $X$ is lost, no matter how small, i.e., $Z_1, \dots, Z_k$ represent a complete partition of $X$.
This is done to save space in the presentation.
Recall, that in the actual execution $Z_i$ have at least $100$ characters and do not necessarily partition $X$.
{
\renewcommand{\sampleinputname}{Laboratory's feedback}
\renewcommand{\sampleoutputname}{Your questions}
\displaysample{../problems/dna-interactive/tests/sample1}
}
\emph{Explanation of the sample 1:}
In this case, $X$ is equal to $E$ and it is broken into substrings $Z_1 = \texttt{AGATAC}$ and $Z_2 = \texttt{ATAGTCA}$.

{
\renewcommand{\sampleinputname}{Laboratory's feedback}
\renewcommand{\sampleoutputname}{Your questions}
\displaysample{../problems/dna-interactive/tests/sample2}
}
\emph{Explanation of the sample 2:}
In this case, $X = \texttt{TGCTAAGCAGATCGA}$ and is different from $E$, it is split into $Z_1 = \texttt{TGCTA}$, $Z_2 = \texttt{AGCAG}$ and $Z_3 = \texttt{ATCGA}$.