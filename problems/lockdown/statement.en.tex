\problemname[stdin/stdout]{Lockdown}
The city where Ashot lives has implemented extremely strict lockdown rules: No one is allowed to go outside for any reason.
There are police officers at many places in the town to enforce the law.
This is terrible news for Ashot, because he walks to his favourite park every day for inspiration to solve the latest programming problems.
So he decided to go to the park without being seen by the police officers.

Ashot has created a map of the city, which is represented by a grid with $R$ rows and $C$ columns.
On his way to the park, Ashot can move from his current location (a cell in the grid) to a horizontally or vertically neighbouring location in one minute.
Alternatively, Ashot can choose to stay at his current location for one minute.

Ashot has discovered that there are $P$ police officers patrolling in the town.
Each officer is assigned a fixed path of locations.
The officer starts from the beginning of his path, and
every minute moves on the next location on the path.
When he reaches the last location of the path, he turns around and follows the path back.
The officer continues his movement like this indefinitely.

Some locations of the city have museums, which are closed, so neither Ashot, nor the police officers can go in them.
A police officer can see Ashot only if they are on the same row or column and there is no museum between them.
Help Ashot to find the least amount of time (in minutes) that he will need to reach the park without being seen by any police officer.

\section*{Input}
The first line of the input contains two space-separated integers $R$ and $C$ ($1 < R, C \leq 60$), the number of rows and columns on Ashot's grid-map of the city.
The second line contains the locations of Ashot's home and his favourite park, separated by a space.
Each location consists of a \texttt{(row column)} pair, written using the parentheses, as shown in the sample input.
The numbering of rows and columns starts from $1$.

$R$ lines follow, each containing exactly $C$ characters.
The character `\texttt{.}' corresponds to a location where both Ashot and the police officers can walk, and `\texttt{\#}' corresponds to museum.

The following line contains a single integer $P$ ($1 \leq P \leq 200$), the number of police officers in the town.
Then $P$ lines follow.
The $i$th line represents the $i$th police officer.
It starts with an integer $k_i$ ($1 \leq k_i \leq 7$), the number of location on the $i$th officer's path.
Then the $k_i$ locations on this officer's path follow as a space-separated list, in the same format as before.

It is guaranteed that the path of each officer is valid: It stays within the city, does not go through a museum and always moves in one of the $4$ directions from any location.
%The  does not go through a museum and stays inside the city.
It is also guaranteed that Ashot's home location and the park do not have a museum.

\section*{Output}
The only line of the output must contain the least amount of time, in minutes, that will take Ashot to reach the park from his home.
If he cannot reach the park, simply output \texttt{-1}.
{
\displaysample{../problems/lockdown/tests/001}
\emph{Note that Ashot is seen by a police officer the moment he arrives to the park on minute 26.
This is ok, because just a brief moment in the park is enough for him to get inspired.}

\displaysample{../problems/lockdown/tests/002}
}
