\problemname[stdin/stdout]{Look-and-Say numbers}
The mathematician John H. Conway (26 December 1937 – 11 April 2020) was famous for many games he invented.
One of those games is Look-and-Say, which is a one-player game that generates a sequence of numbers, starting from $1$.
Each subsequent number is constructed from the previous one using a simple look-and-say rule:
Split the previous number into groups of consecutive equal digits, then read off the groups by saying the number of digits in each group and then the digit itself.
%Read off the digits of the previous member, counting the number of digits in groups of the same digit, and writing the number down as you read it.
It is easier to understand the rule by examples:
\begin{itemize}
    \item $1$ is read off as ``one $1$'' or $11$;
    \item $2223344$ is read off as ``three $2$s, two $3$s, two $4$s'' or $322324$;
    \item $1111332$ is read off as ``four $1$s, two $3$s, one $2$'' or $412312$.
\end{itemize}
Thus, the first numbers of the sequence are
\[
    1, 11, 21, 1211, 111221, ...
\]
Your task is to find $k$th number in the sequence.

\section*{Input}
The only line of the input contains one integer $k$ ($1 \leq k \leq 16)$.
\section*{Output}
The only line of the output must contain the $k$th number in the Look-and-Say sequence.

{
\displaysample{../problems/look-and-say/tests/001}
\displaysample{../problems/look-and-say/tests/004}
\displaysample{../problems/look-and-say/tests/005}
}

%The first number of the sequence is $1$, and each subsequent number is constructed from the previous one using a simple look-and-say rule.
%The rule is easiest to explain by applying it to the number $1113322221122$.
%We group equal and consecutive integers together: $111-33-2222-11-22$, and then create the next number by writing 
%}
