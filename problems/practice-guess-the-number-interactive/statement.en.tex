\problemname[Interactive]{Guess the Number}

This is an interactive problem.
Your program will interact with the one written by the jury using the standard input and output.

The jury’s program is given a hidden integer $h$ between $1$ and $n$ ($1 \leq n \leq 10^6$).
The goal of your program is to guess $h$ in at most $100$ attempts.
You make attempts and the jury's program replies whether your guess is larger, smaller or equal to the given number.

\paragraph{The interaction protocol.}
First, your program needs to read the number $n$ from the first line of the standard input.
Then your program should print its guess attempt $x$ in the standard output.
$x$ must be an integer between $1$ and $n$.
Then your program reads the jury’s program’s response from one line of the standard input.
You may get one of the following inputs:
\begin{itemize}
    \item $-1$ -- this means that $h < x$, i.e., the hidden number is smaller than your guess;
    \item $1$ -- this means that $h > x$, i.e., the hidden number is larger;
    \item $0$ -- this means that the hidden number and your guess are equal.
\end{itemize}
When your program finds the hidden number, it must quit its execution.
Do not forget to flush after each output (see the last page for more details about interactive problems).

\paragraph{Limits.}
The number $n$ is between $1$ and $10^6$, and your program must find the answer in at most $100$ attempts.

\emph{Note, in the examples below, the empty lines are only for presentation.
Your program must not produce them.}
{
\renewcommand{\sampleinputname}{Jury's feedback}
\renewcommand{\sampleoutputname}{Your attempts}
\displaysample{../problems/practice-guess-the-number-interactive/tests/sample1}
\displaysample{../problems/practice-guess-the-number-interactive/tests/sample2}
}
