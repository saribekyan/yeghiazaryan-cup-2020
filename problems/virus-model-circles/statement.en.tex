\problemname[stdin/stdout]{Modelling viruses}
You have been asked to write a program to model the spread of a virus in a population.
As an excellent programmer, you have split the task into simple sub-tasks.
In your current sub-task you model the movement of two people, Alice and Bob.
Your goal is to determine if they violate the social-distancing rules, and if so, when.

Alice and Bob are represented as points moving in the two dimensional space.
Alice starts her movement at point $A$ at time $0$ and
moves with a constant velocity given by a vector $v_A$.
You have the same information about Bob.
Both Alice and Bob move for exactly $T$ seconds, and then stop.

Alice and Bob violate the social-distancing rule if their distance is $2$ metres or less.
Write a program that determines the earliest time when they violate the rule, if they ever do.

\section*{Input}
The first line of the input contains a single integer $T (1 \leq T \leq 10^5)$, the length of the time interval of interest.
The second line contains the information about Alice, four space-separated integers $x, y, v_x, v_y$,  which are all at most $10^5$ in absolute value.
The starting position of Alice is $A = (x, y)$ and her velocity is given by the vector $v_A = (v_x, v_y)$.
The third line contains the information about Bob, in the same format and satisfying the same conditions.
All the values are in standard units, i.e., metres and seconds.

\section*{Output}
The only line of the input should have the first time, in seconds,
when Alice and Bob are violating the social-distancing rule.
If this never happens, simply output $-1$.
Your output should have at least $5$ digits after the decimal point.
