\problemname[stdin/stdout]{Roman road}
% Author: Hayk Saribekyan
The towns of Cowford and Wentbridge are connected by an old one-way Roman road.
The roads that Romans built are very straight, so we will represent the road between Cowford and Wentbridge as a straight line of length $L$ meters.

The Roman roads were not designed for cars, so some cars cannot pass one another.
In particular, if a faster car of width $w_F$ approaches a slower car of width $w_S$, then the faster car can only overtake if $w_F + w_S \leq W$,
where $W$ is the width of the road.
Otherwise, the faster car is stuck behind the slower one and the two cars have to drive at the same (slower) speed.
Since the cars are not very long, we will assume that the cars in such a situation arrive to Wentbridge at the same time, i.e., we assume that the cars are points.

On the weekend, the $N$ residents of Cowford go to Wentbridge in their cars.
Resident $i$ leaves Cowford at time $t_i$, and moves at speed $s_i$ meters per second towards Wendbridge, along the Roman road.
Additionally, the width of the car of resident $i$ is $w_i$ microns.

The residents have asked you to write a program that, for each resident, determines the time when they arrive to Wentbridge.

\section*{Input}
The first line of the input contains three integers: $N$ ($1 \leq N \leq 10^5$), the number of Cowford residents; $L (1 \leq L \leq 10^9)$ the length of the Roman road in meters; $W (1 \leq W \leq 10^7)$ the width of the roads in microns.

Line $i$ of the input after that contains three integers:
$t_i (0 \leq t_i \leq 10^9)$\htodo{given the time or the initial coordinates}, the time resident $i$ left Cowford;
$s_i (1 \leq s_i \leq 10^9)$, the speed at which that resident travels in meters per second;
$w_i$ ($1 \leq w_i \leq W)$, the width of the $i$th resident's car in microns.
\htodoi{improve sample tests}

\section*{Output}
The $i$th line of the output contains on real number
-- the time, in seconds, when the $i$th resident of Cowford arrives to Wentbridge.
Print the answers with $2$ digits after the decimal point.